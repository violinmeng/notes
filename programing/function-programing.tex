% UTF-8 encoding
% Compile with latex+dvipdfmx, pdflatex, xelatex or lualatex
% XeLaTeX is recommanded
%!TEX program = xelatex
% Switch Compiler to XeLaTeX in VSCode:
% Add the following line to settings.json:
% `"latex-workshop.latex.recipe.default": "latexmk (xelatex)",`
% more details:
% https://stackoverflow.com/questions/56109128/enable-xelatex-in-latex-workshops-for-visual-studio-code

\documentclass[UTF8]{ctexart}

\usepackage{amsmath}
\usepackage{amssymb}
\usepackage{graphicx}

\usepackage{listings}
\usepackage{xcolor}
\usepackage{geometry}

\geometry{
  a4paper, % 纸张大小
  left=2.5cm, % 左边距
  right=2.5cm, % 右边距
  top=2cm, % 上边距
  bottom=2cm, % 下边距
}

\lstset{
  language=Java,
  basicstyle=\ttfamily\small,
  keywordstyle=\color{blue},
  commentstyle=\color{gray},
  stringstyle=\color{orange},
  breaklines=true,
  frame=single, % 添加单行边框
  showstringspaces=false, % 不显示字符串中的空格
  columns=flexible
}

\title{函数式编程概念梳理}
\author{E. W.}
\date{\today}

\begin{document}

\maketitle
\tableofcontents

\section{Purity \& Side Effects}
\subsection{The Pros of Purity}
\begin{itemize}
    \item Cachable
    \item Portable \& Self-documenting
    \item Testable
    \item Reasonable
    \item Parallelizable
\end{itemize}

\section{Currying}

\begin{equation}
    f(x,y,z) \implies  f(x)(y)(z)
\end{equation}

\begin{lstlisting}
    const add = x => y => x + y;
    const increment = add(1);
    const addTen = add(10);

    increment(2); // 3
    addTen(2); // 12
\end{lstlisting}

\section{Composing}

\begin{equation}
    f(x) \quad g(x) \implies f(g(x)) \label{eq:compose}
\end{equation}
Reference Equation~\eqref{eq:compose}.

\begin{lstlisting}
    const compose = (f, g) => x => f(g(x));

    const add1 = x => x + 1;
    const mult2 = x => x * 2;

    const add1Mult2 = compose(mult2, add1);
    add1Mult2(5); // 12
\end{lstlisting}

\textbf{Point-free style}
\begin{lstlisting}
    // not pointfree because we mention the data: word
    const snakeCase = word => word.toLowerCase().replace(/\s+/ig, '_');

    // pointfree
    const snakeCase = compose(replace(/\s+/ig, '_'), toLowerCase);
\end{lstlisting}

\subsection{Category Theory}
Definition of Category: A collection with the following components:
\begin{itemize}
    \item A collection of \textit{Objects}
    \item A collection of \textit{Morphisms}
    \item A notion of \textit{Composition} on the morphisms
    \item A distinguished morphism called \textit{Identity}
\end{itemize}

In the context of programming,
the \textit{Objects} are types,
the \textit{Morphisms} are functions,
\textit{Composition} is function composition,
and the \textit{Identity} is the identity function.

\begin{lstlisting}
    // identity
    compose(id, f) === compose(f, id) === f;
    // true
\end{lstlisting}

In Haskell, the compose operator is represented by a dot:
\begin{lstlisting}
    (.) :: (b -> c) -> (a -> b) -> a -> c
\end{lstlisting}

\section{Hindley-Milner Type System}

The explanation of \textit{Free as in Theorem}:
Write down the definition of a polymorphic function on a piece of paper.
Tell me its type, but be careful not to let me see the function's definition.
I will tell you a theorem that the function satisfies.

\section{Functor}

Definition of Functor:
A type that implements \texttt{map} and obeys some laws. And the laws are:
\begin{enumerate}
    \item Identity: \texttt{fx.map(x => x) === fx}
    \item Composition: \texttt{fx.map(f).map(g) === fx.map(x => g(f(x))}
\end{enumerate}

\section{Example Figure}
\begin{figure}[htbp]
    \centering
    \includegraphics[width=0.5\textwidth]{example-image-a} % Replace with your image file
    \caption{An example figure.}
    \label{fig:example}
\end{figure}
See Figure~\ref{fig:example} for an illustration.

\end{document}